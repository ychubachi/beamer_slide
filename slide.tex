\documentclass[dvipdfm]{beamer}
\AtBeginDvi{\special{pdf:tounicode 90ms-RKSJ-UCS2}}
\usepackage{atbegshi} % しおりの和文対応
\usepackage{minijs}

\renewcommand{\kanjifamilydefault}{\gtdefault} % 本文ゴシック体

%%%%%%%%%%%  theme  %%%%%%%%%%%
%\usetheme{Madrid}
%\usetheme{AnnArbor}
%\usetheme{Antibes}
%\usetheme{Barkeley}
%\usetheme{Bergen}
\usetheme{Berlin}
%\usetheme{Boadilla}
%\usetheme{CambridgeUS}
%\usetheme{Copenhagen}
%\usetheme{Darmstadt}
%\usetheme{default}
%\usetheme{Dreson}
%\usetheme{Frankurt}
%\usetheme{Goettingen}
%\usetheme{Hanover}
%\usetheme{Ilmenau}
%\usetheme{JuanLesPins}
%\usetheme{Luebeck}
%\usetheme{Malmoe}
%\usetheme{Marburg}
%\usetheme{Montpellier}
%\usetheme{PaloAlto}
%\usetheme{Pittsburgh}
%\usetheme{Rochester}
%\usetheme{Singapore}
%\usetheme{Szeged}
%\usetheme{Warsaw}


%%%%%%%%%%%  inner theme  %%%%%%%%%%%
%\useinnertheme{default}
\useinnertheme{circles}
%\useinnertheme{inmargin}
%\useinnertheme{rectangles}
%\useinnertheme{rounded}


%%%%%%%%%%%  outer theme  %%%%%%%%%%%
%\useoutertheme{default}
%\useoutertheme{miniframes}
%\useoutertheme{infolines}
%\useoutertheme{miniframes}
\useoutertheme{smoothbars}
%\useoutertheme{sidebars}
%\useoutertheme{split}
%\useoutertheme{shadow}
%\useoutertheme{tree}
%\useoutertheme{smoothtree}


%%%%%%%%%%%  color theme  %%%%%%%%%%%
%\usecolortheme{structure}
%\usecolortheme{sidebartab}
%\usecolortheme{albatross}
%\usecolortheme{beetle}
%\usecolortheme{dove}
%\usecolortheme{crane}
%\usecolortheme{fly}
%\usecolortheme{seagull}
%\usecolortheme{wolverine}
%\usecolortheme{beaver}
%\usecolortheme{lily}
%\usecolortheme{orchid}
%\usecolortheme{rose}
%\usecolortheme{whale}
%\usecolortheme{seahorse}
%\usecolortheme{dolphin}


%%%%%%%%%%%  font theme  %%%%%%%%%%%
\usefonttheme{professionalfonts}
%\usefonttheme{default}
%\usefonttheme{serif}
%\usefonttheme{structurebold}
%\usefonttheme{structureserif}
%\usefonttheme{structuresmallcapsserif}


%%%%%%%%%%%  degree of transparency  %%%%%%%%%%%
%\setbeamercovered{transparent=30}


%%%%%%%%%%%  numbering  %%%%%%%%%%%
%\setbeamertemplate{numbered}

%\setbeamertemplate{navigation symbols}{}

\AtBeginSection[]
{\begin{frame}{発表の流れ}\tableofcontents[currentsection]\end{frame}}


\title[題目(下部)]{題目}
\author[氏名(下部)]{氏名}
\institute[所属(下部)]{所属}
\date{June 25th}
%\date{\today}

\begin{document}
\maketitle

\section{あんなこんな}
%%%%%%%%%%%% シート(1枚目) %%%%%%%%%%%%
\subsection{タイトル}
\begin{frame}[t]
 \frametitle{タイトル}

\[
  \int \sqrt{x^2+A}\, dx
  =\dfrac{1}{2}\left(x\sqrt{x^2+A}+A\log\big|x+\sqrt{x^2+A}\big|\right)
\]

 \begin{enumerate}
   \item 1
   \item 2
 \end{enumerate}

\end{frame}
%%%%%%%%%%%% シート(1枚目) %%%%%%%%%%%%


%%%%%%%%%%%% シート(2枚目) %%%%%%%%%%%%
\subsection{ブロック}
\begin{frame}[c]
 \frametitle{ブロック}
 \begin{block}{基本ブロック}
  基本的なブロックです.
 \end{block}
\end{frame}
%%%%%%%%%%%% シート(2枚目) %%%%%%%%%%%%

\section{次のセクション}
%%%%%%%%%%%% シート(3枚目) %%%%%%%%%%%%
\subsection{オーバーレイ}
\begin{frame}
 \frametitle{オーバーレイ(数字)}
 \begin{itemize}
   \item<1-> 1枚目
   \item<2-3> 2枚目
   \item<3> 3枚目
   \item<4> 4枚目
 \end{itemize}
\end{frame}
%%%%%%%%%%%% シート(3枚目) %%%%%%%%%%%%


%%%%%%%%%%%% シート(4枚目) %%%%%%%%%%%%
\subsection{色付け}
\begin{frame}[t, fragile]
 \frametitle{色付け}
  {\color{red} red}(\alert{alert}),
  {\color{blue} blue}(\structure{structure}),
  {\color{green} green},
  {\color{cyan} cyan},
  {\color{magenta} magenta},
  {\color{yellow} yellow},
  {\color{black} black},
  {\color{darkgray} darkgray},
  {\color{gray} gray},
  {\color{lightgray} lightgray},
  {\color{orange} orange},
  {\color{violet} violet},
  {\color{purple} purple},
  {\color{brown} brown},

\end{frame}
%%%%%%%%%%%% シート(4枚目) %%%%%%%%%%%%





%%%%%%%%%%%% シート(5枚目) %%%%%%%%%%%%
\begin{frame}[t]
 \frametitle{特殊効果}
  \transboxin<1>
  \transboxout<2>
   \begin{enumerate}
  \item<1> 
   {\huge transboxin
   \[\int_{0}^{1}\tan x\, dx=\dfrac{\pi}{4}\]
   }
  \item<2>
   {\huge transboxout
   \[\int_{0}^{1}\tan x\, dx=\dfrac{\pi}{4}\]
   }
   \end{enumerate}
\end{frame}
%%%%%%%%%%%% シート(5枚目) %%%%%%%%%%%%




%%%%%%%%%%%% シート(6枚目) %%%%%%%%%%%%
\begin{frame}[t]
 \frametitle{コラム}
  \begin{columns}[t]
   \begin{column}{.3\textwidth}
    左側に説明などなど
   \end{column}
   \begin{column}{.4\textwidth}
    \begin{alertblock}{ブロック}
     ブロックなども当然入れられます.
    \end{alertblock}
   \end{column}
  \end{columns}
\end{frame}
%%%%%%%%%%%% シート(6枚目) %%%%%%%%%%%%


\end{document}
